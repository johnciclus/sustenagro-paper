%% LyX 2.1.4 created this file.  For more info, see http://www.lyx.org/.
%% Do not edit unless you really know what you are doing.
\documentclass[english]{llncs}
\usepackage[T1]{fontenc}
\usepackage[latin9]{inputenc}

\makeatletter
%%%%%%%%%%%%%%%%%%%%%%%%%%%%%% User specified LaTeX commands.
\usepackage{babel}

\makeatother

\usepackage{babel}
\begin{document}

\title{Ontologies and Domain Specific Language to Support the Decision Making
in Agricultural Sustainability, SustenAgro case.}


\author{John Garavito \and Dilvan Moreira \and Katia Regina \and Ivo Pierozzi}

\maketitle
Institute of Mathematical and Computer Sciences , S�o Paulo University,\\
 Avenida Trabalhador S�o-carlense, 400 S�o Carlos - SP, Brazil\\

\begin{abstract}
Agricultural systems have the need of to measure their sustainability,
an approach to meet this goal is the use of Indicators of Sustainable
Development (ISDs), which was used to define the SustenAgro method,
which purpose is provide sustainability assessment in sugar-cane production
systems at the center-south of Brazil. 

This paper we present the representation of this method trough of
two ontologies, the ontology and the decision support ontology which
represent the domain knowledge and an Domain Specific Language (DSL)
that manages the knowledge, these elements support the software tool
entitled SustenAgro presented in this paper and makes it capable to
work with Semantic Web architecture. \keywords{Indicators of Sustainable
Development, Sustainability Assesment, Decision Making, Sustainable
Ontology, User Interface Ontology, Domain Specific Language} 
\end{abstract}


\section{Introduction}

The agricultural sustainability assessment integrates the environmental,
social and economic dimension of a production system, in the SustenAgro
project the sugar-cane production system was modeling and systemized,
allowing improve the sustainability in the farms and plants related
with sugar-cane production as well as others sub products like sugar,
bioethanol, yeast, bagasse and electric energy. 

Sustainable Development in this context, is the base to serve the
present needs without compromise the future generations to supplement
their needs\cite{brundtland1987our}, taking into account the sustainability
dimensions: environmental, social and economic \cite{alkan2009goal}.
Due to complexity of this process is necessary to define metrics that
quantify the sustainability.

About the sustenagro project was led and executed by Embrapa Environment,
and generated the base knowledge and the indicators which support
the assessment process and the development of SustenAgro Software, 

The objetive of this paper is presents the semantic web based solution
for support the sustainability assessment process in sugarcane and
represent the experts knowledge in a computable format. 

Indicators represent the critic information about the system, which
is possible to take immediate problems or prevent future problems,
the base of this paper was an research who identify and define the
indicators of sustainability\cite{brunooliveira2013}, ), this indicators
was used as conceptual raw material.\textit{ (os indicadores representam
a informa��o cr�tica sobre o sistema, o que � poss�vel resolver os
problemas imediatos ou evitar problemas futuros, a base deste trabalho
foi uma pesquisa que identificou e definiu os indicadores de sustentabilidade
(oliveira cardoso, 2013), estes indicadores foram utilizados como
conceitual mat�ria-prima.) }

We utilize an agile method to create the ontology, we adapt this method
to facilitate to make this work with a team. To create we make daily
reunions and adapt the knowledge to a conceptual map. We propose work
this way because the experts have no knowledge about the tools to
create ontologies and to make an easy communication.

This ontology is a part of an project called SustenAgro, this project
proposes the creation of a computational system whose objective is
evaluate the sustainability in production of sugarcane and soy in
center-south of Brazil. Is important to create an ontology because
of the diversity of team; it\textquoteright s support to equalize
the knowledge of experts; and to support the computational system
whose will be based on semantic web technologies. 

Due to the diversity and the changing nature of the indicators, the
construction of a methodology for the evaluation sustainability and
the requirement to have a massive system of data collection, was necessary
to analyze the technological possibilities to provide an architecture
and information retrieval system to deal with said requirements. It
was decided to implement an information retrieval system based on
triplestore, which needs the development of a titled ontology.

The agricultural systems involve environmental, social and economic
aspects, which require a full understanding and inter-disciplinary
field, is given complexity of the phenomenon arises the need to maintain
a conceptual basis to organize and represent the concepts used by
the team of experts and the systems computer.

Specifically the area of knowledge of SustenAgro Ontology seeks support
concepts that are constantly changing, for example in the process
of \textquotedbl{}sustainability assessment\textquotedbl{} the indicators
and indexes are continuously redefined.

The hypothesis that we want to validate is if SustenAgro Ontology
will support conceptual and technologically representing the domain
that is in continuous change, the methodology for evaluating sustainability
and a software system information retrieval.

O fornecimento de um metodo de avalia��o da sustenabilidade em agricultura
� uma necessidade latente. Pelo qual pesquisadores da Embrapa Meio
Ambiente desenvolveram um metodologia de avalia��o da sustentabilidade
intitulada 'Metodo SustenAgro' focado em uma cultura e regiao especifica
do Brasil que permita integrar o conhecimento relacionado com a sustentabilidade
e fornecer uma medi��o dele para ter uma metrica da sustentabilidade
e gerar recomenda��es para veicular estrategias de melhora da sustentabilidada,
al�m de posibilitar a inform��o base para a formula��o de politicas
publicas.

Embrapa meio ambiente

Suportar um metodo de avalia��o da sustentabilidade 

farmers and stakeholders


\section{The problem}


\section{Architecture.}

O sistema software abrange tecnologias da web semantica que permitem
representar o conhecimento de uma forma estructurada por meio de ontologias
que


\section{Methodology.}

The development of ontology is responsive type by techniques of rapid
prototyping of coverage and increased complexity, starting with the
most relevant components of the model to the experts and incorporating
each one of the other components by validations, this method was cyclical
obtaining in each cycle ontology prototypes.

Among them are the conceptual maps that allow a focused communication
in the field of specialist, in order to represent knowledge of a visual
way and we also have the computational models in semantic web formats
that allow communication with experts in modeling of knowledge.

After having the conceptual model well-defined, modelers represented
this model in OWL-DL standard and popular data storage system, after
that the expert built questions that were asked in the system, and
it generated the expected results, then follows a validation phase,
integration and adjustment, which ended with a reliable prototype
that represents an ontology sector, this process will be repeated
until you have all sectors of interest and the required integrity.

The methodology includes the following steps that will represent the
domain knowledge, the process corresponds to a cyclical methodology,
which will be added sectors according to the maturation of the terms
and the need of the information contained:
\begin{enumerate}
\item Definition of entities
\item Definition of ratings
\item Definition of semantic relations
\item Definition the rules and axioms
\item Implementation in OWL
\item Instantiation of individuals
\item Construction of questions
\item Validation through consultation
\item Adjustments and integration
\end{enumerate}
This process is not necessarily linear, as moderators can return to
the previous steps as necessary..

The figure 1 represents how is the methodology and how we work cyclic
and circular with it.

The modules that were addressed in the modeling are, in order of development:
\begin{enumerate}
\item Module attributes and data from production plants of sugarcane
\item Module assessment of sustainability indicators in sugarcane systems
\item Module representation of sustainability assessment methodology
\item Module georeferencing connection with the supply of natural resources
data.
\end{enumerate}

\section{Development and Tests}


\section{Related work}


\subsection{Agricultural Area}

This part describes the pappers that were relevant to the knowledge
in agriculture.

It is necessary for a standardized ontology using techniques which
make it multilingual so that it can be used without the need for adjustments
to researchers from anywhere.

The papper\cite{lauser2006agrovoc} describes a technique that uses
the AGROVOC multilingual Theasaurus for conversion into an ontology
and also covers the concepts of agriculture that are a focus of the
work, so we analyzed the techniques used to adapt to our ontology.


\subsection{Sustainability Area}

This part describes the pappers that were relevant to the knowledge
in the area of sustainability.

For evaluation of sustainability requires working with indicators
that measure how each practice is sustainable. In the paper\cite{brilhante2006information}
describes an sustainable analysis framework of Amazonas state and
is developed an ontology of these indicators that work in conjunction
with the framework. The last foundation for this work was extremely
important for the development done. Our ontology is based on agricultural
system, the focus on sustainability help to amplify the knowledge
and how to make it computational.

The ontology must be flexible to some extent to accept changes according
to the need of the specialist. The concepts presented by\cite{kraines2011system}
gave an idea of how to make flexible ontology in the area of sustainability
so that it can support new concepts and still not miss the formality
or allow the ambiguous concepts. In this way, we adapt this concepts
and make it in ours.

To describe the sustainability of a clearer and adaptively is important
to understand several indicators that are used in several different
places.\cite{wilson2007contrasting} makes a comparison between sustainability
indicators going through in most of it, analyzing global metrics and
interpreting sustainability among nations, through a clear notion
of sustainability to support knowledge used in the ontology. 


\subsection{Ontologies Area}

In this part of the pappers that were relevant to the knowledge in
the area of development of ontologies in general are described.

For a good ontology is important that the concepts are well defined
and for this and necessary to use techniques to detect semantic conflicts.\cite{alcaraz2010detection}
defines how to assemble senarios in order to detect conflicts or problems
that may have been made in the creation of ontology, the techniques
described can be considered as a kind of reasoning that detects situations
that was not explicit in the ontology. With this and other pappers
related to sustainability we can disambiguate and ensure the concepts
are consistent and follow the expert needs.

For development of an ontology is important to understand the concepts
involved and needs to create an ontology.\cite{wen2007event} describes
about ontologies in computer science and generalized terms that are
needed for development of ontologies, these terms were extremely important
for all stages of creation and analysis of ontology. We do patterns
in links between classes to try make more consistent the ontology.


\subsection{Conceptual basis of ontology}

\cite{brunooliveira2013} was taken as the basis of knowledge and
was chosen by the domain expert. Was taken all the knowledge base
that was used as the indicators and the needs of each and still had
part of integrated sustainability directly with the part of agriculture
geared to the location of the state of S�o Paulo. It was analyzed
with all the parts described in the dissertation by selecting indicators
previously validated by a group of experts. 


\subsection{Related Technologies}

In order to eliminate ambiguity in semantic understanding and making
mining the relationship between the concepts of knowledge in agriculture,
should be combined with the knowledge among heterogeneous databases.

It proposes a Agricultural Knowledge Grid\cite{cuiping2013agricultural}
which was built with three layers \textquotedbl{}Resource Layer\textquotedbl{},
\textquotedbl{}Semantic Layer\textquotedbl{} and \textquotedbl{}User
Layer\textquotedbl{}, has been applied to \textquotedbl{}Semantic
Extension on Retrieval\textquotedbl{}, \textquotedbl{}Knowledge links\textquotedbl{}
and \textquotedbl{}Experience\textquotedbl{} to deepen \textquotedbl{}Agricultural
Knowledge Services\textquotedbl{}.

AGROVOC is a controlled vocabulary that covers all areas of interest
FAO and consists of 32,000 concepts available in 21 languages, this
tool is used by researchers, librarians and information managers to
index, retrieve and organize data in agricultural information systems.

\cite{lauser2006agrovoc} is an initiative that serves as a reference
in structuring and standardization of agricultural terminology in
multiple languages for use systems in agriculture, the purpose of
this technology is to achieve more interoperability between agricultural
systems.


\section{Results}


\section{Discussion}


\section{Conclusions}

\bibliographystyle{plain}
\bibliography{sustenagro}

\end{document}
